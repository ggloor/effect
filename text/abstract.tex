\abstract{
\noindent\textbf{Motivation:} High throughput sequencing is analyzed using a combination of null hypothesis significance testing and ad-hoc cutoffs. This framework is strongly affected by sample size, and is known to be irreproducible in underpowered studies, yet no suitable non-parameteric alternative has been proposed. \\
\textbf{Results:} Here we present implementations of non-parametric standardized median effect size estimates,  $\mathcal{E}_{d}$, for high-throughput sequencing datasets. Case studies are shown for transcriptome and amplicon-sequencing datasets.  The  $\mathcal{E}_{d}$ statistic is shown to be more reproducible and robust than p-values and requires sample sizes as small as 5 to reproducibly identify  differentially abundant features.\\
\textbf{Availability:} Source code and binaries freely available at:\\
 https://bioconductor.org/packages/ALDEx2.html, omicplotR, and\\  https://github.com/ggloor/CoDaSeq.\\
\textbf{ggloor@uwo.ca}{ggloor@uwo.ca}\\
\textbf{Supplementary information:} Supplementary data and code will be available when published.
