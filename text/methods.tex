We used  simple simulated datasets to determine baseline characteristics in a number of different distributions. Then we use the data  from a highly replicated RNA-seq experiment \citep{Schurch:2016aa} and examined 100 random subsets of the data with between 2 and 40 samples in each group. For each random subset we collected the set of features that were called as differentially abundant at thresholds of $\mathcal{E}_{d} \ge 1$, or with an expected Benjamini-Hochburg adjusted p-value of $\le 0.1$ calculated using either the parametric Welch's t-test, or the non-parametric Wilcoxon test in the ALDEx2 R package. These are output as `we.eBH' and 'wi.eBH' by the ALDEx2 tool. These were compared to a `truth' set determined by identifying those features that were identified in all of 100 independent tests of the full dataset with outliers removed using the same tests and cutoffs. Note that this is simply a measure of consistency and is congruent with the approach taken in \citep{Schurch:2016aa}.

The median values and 99\% confidence intervals for the number of true positives, false positives, true negatives and false negatives were tabulated and plotted in Fig. 1.  
